\chapter{Avances}

Durante esta primera fase del proyecto se han cumplido varios de los objetivos plantados en el proyecto. Por el \'area de software hemos podido ya implementar la parte del simulador de ARM realizando pruebas y logrando simulación de ciertas aplicaciones generadas por el compilador y el emsamblador de GNU.\medskip

Se han implementado las bibliotecas din\'amicas que permiten programar de una manera muy sencilla dispositivos de ARM.\medskip

Para facilitar el uso del simulador se ha decidido empaquetar con autotools, lo que le permite que los desarrolladores de UNIX se sientan familiarizados con la manera de configurar y compilar nuestro simulador, autotools ha permitido generar a su vez de una manera muy sencilla paquetes binarios para algunas distribuciones de Linux.\medskip

Por la parte del hardware se ha trabajado en el dise\~no de la tarjeta, la mayor parte de este trabajo se han buscado los componentes que pueden cumplir con los requerimientos que tenemos y un diseño en esquem\'atico del hardware. Se han comprado y probado algunos de ellos. Cabe destacar el trabajo de probar los componente de FTDI, y el dispositivo de video llamado: Picaso.\medskip

A continuaci\'on se explican de una manera m\'as detalladas los avances del proyecto.

